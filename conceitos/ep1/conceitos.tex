\documentclass[brazil,times]{abnt}
\usepackage[T1]{fontenc}
\usepackage[utf8]{inputenc}
\usepackage{url}
\usepackage{graphicx}
\usepackage[pdfborder={0 0 0}]{hyperref}
\usepackage{amssymb}
\makeatletter
\usepackage{babel}
\makeatother

\begin{document}


\autor{Arthur Branco Costa - 7278156 \\
      Daniel Paulino Alves - 7156894 \\
      Felipe Yamaguti - 7295336 \\
      Pedro Paulo Vezzá Campos - 7538743 \\
      Thiago Tatsuo Nagaoka - 7289197}

\titulo{Conceitos da Linguagem Modula-3}

\comentario{Primeiro exercício-programa apresentado para avaliação na disciplina
MAC0316, do curso de Bacharelado em Ciência da Computação, turma 45, da
Universidade de São Paulo, ministrada pela professora Ana Cristina Vieira de
Melo.}

\instituicao{Departamento de Ciência da Computação \par Instituto de Matemática
e Estatística \par Universidade de São Paulo}

\local{São Paulo - SP, Brasil}

\data{\today}

\capa

\folhaderosto

\tableofcontents

\chapter{Introdução}
Programas são feitos com o intuito de, através de um conjunto de pré-condições, manipular os dados de entrada ao longo de processos, de forma a produzir um resultado útil e eficaz, desde que elas sejam respeitadas.

Eles podem diferir em diversas características: paradigma, lógica computacional, eficiência, legibilidade, simplicidade, entre outros. No entanto, um fator poderia ser classificado como principal: a linguagem de programação na qual eles são escritos. Com muitos de seus conceitos já pré-definidos pelos seus projetistas, e a possibilidade de que o programador desenvolva outros, as linguagens de programação são altamente relacionadas aos fatores mencionados.

Assim, a escolha adequada da linguagem determina a expressividade e facilita a criação de um bom programa. Desta forma, é necessário que o programador tenha um bom conhecimento da linguagem escolhida, para que possa obter seu maior aproveitamento.

Este trabalho visa analisar as propriedades presentes na linguagem Modula-3
e os conceitos a serem apresentados decorrem de tal necessidade. \cite{ana:livro} \cite{cardelli:Modula3LanguageDefinition}
 
\chapter{Tipos primitivos e compostos}
Para alcançar o seu objetivo, um programa deve lidar com valores. Para isso, é preciso que operações que os relacionem sejam definidas. A fim de evitar a repetição excessiva dessas definições, os valores são agrupados em conjuntos de mesma espécie com propriedades próprias: os tipos.

Divididos em tipos primitivos e compostos.  

\section{Tipos primitivos}
Ocorrem caso seus valores não possam ser subdivididos em elementos pertencentes ao mesmo conjunto. Ou seja, as partes podem existir, mas não são semanticamente relacionadas.

Eles dividem-se em Integer, Cardinal, Boolean, Char, Enumerated e Reference.

%%%%%%%%%%%%
%Ordinais (Intervalos de representação)
% - INTEGER: Todos os números representáveis pela implementação (Similar a [MIN_INT, MAX_INT])
% - CARDINAL: Similar a [0, MAX_INT]
% - BOOLEAN: Enumeração {FALSE, TRUE}
% - CHAR: Enumeração com pelo menos 256 elementos
% - {A, B, C}
% - [T1.A..T1.C]

%Ponto flutuante:
% - REAL
% - LONGREAL
% - EXTENDED

%Array:
% - VAR a ARRAY Ordinal of REAL;

%Records:
%	Idem structs

%Set:

%Referências:
% - traced: Com coleta de lixo
% - untraced: Sem coleta de lixo

%Exemplos de código:
% - http://rosettacode.org/wiki/Category:Modula-3

%%%%%%%%%%%%


\subsection{\texttt{INTEGER}}
Lorem ipsum
\subsection{Cardinal}
Lorem ipsum
\subsection{Boolean} 
\subsection{Char} 
\subsection{Enumerated} 
\subsection{Reference} 

\section{Tipos Compostos}
Este tipo diz respeito a um conjunto de valores cujos elementos podem ser separados em partes menores e mais simples, que ainda pertencem ao todo.

Dividem-se em Array, Record, Set e Object.

\subsection{Array} 
\subsection{Record} 
\subsection{Set} 
\subsection{Object} 

\chapter{Variáveis simples e compostas}
Programas manipulam e armazenam valores através de variáveis. Elas são caracterizadas por um identificador, um endereço de memória e um valor associado.

Do ponto de vista de seus armazenamento e acesso são classificadas em simples e compostas.

\section{Variáveis simples}
Armazenam valores que são acessados de maneira pontual, ou seja, a modificação e o acesso de seus valores são feitos como um todo.

a: INTEGER

\section{Variáveis compostas}
Variáveis que armazenam valores que podem ser acessados e modificados individualmente. Isto é, o tratamento de seus elementos é seletivo.


\chapter{Variáveis quanto a sua existência}
Variáveis também podem ser caracterizadas segundo sua existência. Tal classificação abrange variáveis globais, locais, \textit{heap} e persistentes.

\section{Variáveis globais}
São variáveis que existem antes do programa ser executado e permanecem ativas até o seu término.

\section{Variáveis locais}
Variáveis cuja existência é delimitada pelo escopo no qual está inserida. Logo, mostram-se ativas desde o momento de sua declaração até o fim do escopo.

\section{Variáveis \textit{heap}}
Ao contrário das anteriores, a existência destas é condicionada à alocação e desalocação dinâmicas de memória. Portanto, elas são criadas durante o tempo de execução do programa e encerradas após a desalocação de sua memória ocupada, ou no final, prática não recomendada por gerar vazamento de memória, pois cabe ao sistema operacional forçar a desalocação desse espaço.

\section{Variáveis persistentes}
Estas variáveis destacam-se das demais por serem as únicas que, com o encerramento do programa, não são destruídas. Desta maneira, suas informações são armazenadas para uso posterior.


\chapter{Forma e tempo de vinculação de tipos às variáveis}


\chapter{Sistemas e verificação de tipos utilizados na linguagem}


\chapter{Abstrações}

\bibliographystyle{abnt-num}
\bibliography{bibliografia}

\end{document}
