\documentclass[12pt, a4paper, oneside]{article}

\usepackage[utf8]{inputenc}
\usepackage[portuguese]{babel}
\usepackage[T1]{fontenc}
\usepackage[left = 2cm, right = 2cm, top = 2cm, bottom = 2cm]{geometry}
\usepackage{indentfirst}
\usepackage{listings}

\lstset{
language=C,
tabsize=2,
inputencoding=utf8,
basicstyle=\small,
showspaces=false,
showstringspaces=false,
showtabs=false,
% columns=fullflexible
}

\title{\huge{MAC0328 - Algoritmos em Grafos - Lista 1}}

\author{Pedro Paulo Vezzá Campos - 7538743}

\date{\today}

\begin{document}

\maketitle

\section*{Questão 1} Sejam \texttt{d[w]}, \texttt{f[w]} e \texttt{parnt[w]}
respectivamente os tempos de descoberta, finalização e o vértice pai na
arborecência DFS do vértice w.

A partir do momento que o algoritmo de busca em profundidade descobre o vértice
u temos algumas possibilidades:
\begin{description}
  \item [O vértice v ainda não foi descoberto.] \hfill
  	\begin{enumerate}
  		\item Caso o algoritmo descubra v através de u $\rightarrow$ v temos que
  			\texttt{d[u]} < \texttt{d[v]} e \texttt{parnt[v]} = u. Ainda, vale
  			\texttt{f[v]} < \texttt{f[u]} pela definição de DFS. Portanto u
  			$\rightarrow$ v é arco de arborecência. Automaticamente o arco v
  			$\rightarrow$ u será de retorno pela definição de arco de retorno
  			(\texttt{d[u]} < \texttt{d[v]} < \texttt{f[v]} < \texttt{f[u]}).
		\item Caso o algortimo descubra v por outro caminho que não u $\rightarrow$ v
			temos novamente que \texttt{d[u]} < \texttt{d[v]} e \texttt{f[v]} <
			\texttt{f[u]} pois v é um descendente de u. Neste caso, no entanto,
			\texttt{parnt[v]} $\ne$ u pois não foi utilizado o arco u $\rightarrow$ v para
			descobrir v. Isso denota que u $\rightarrow$ v é um arco descendente.
			Novamente, v $\rightarrow$ u será arco de retorno pela definição de arco de
		retorno.
	\end{enumerate}
\end{description}

\begin{description}
  \item [O vértice v já foi descoberto mas ainda não foi fechado.] \hfill
	\begin{enumerate}
	  \item Caso o vértice
u tenha sido descoberto através do arco v $\rightarrow$ u estamos em um caso análogo ao
1.1. Vale que \texttt{d[v]} < \texttt{d[u]} e \texttt{parnt[u]} = v e pela definição de DFS vale \texttt{f[u]} <
\texttt{f[v]}. Com isso concluímos que v $\rightarrow$ u é da arborecência e que u $\rightarrow$ v é de
retorno.
	  \item Caso o algortimo descubra v por outro caminho que não v $\rightarrow$ u
estamos em um caso análogo ao 1.2. Vale novamente que \texttt{d[v]} < \texttt{d[u]} já que u é
descendente de v. Mas \texttt{parnt[v]} $\ne$ v pois não utilizamos v
$\rightarrow$ u para descobrir u. Concluímos assim que v $\rightarrow$ u é um arco
descendente. Novamente, u $\rightarrow$ v será arco de retorno pela definição de arco de retorno.
	\end{enumerate}
\end{description}

\begin{description}
  \item [O vértice v já foi descoberto e fechado. (Impossível!)] \hfill
  
  Isso implicaria que todos os vértices atingíveis a partir de v já teriam sido
  descobertos e fechados, inclusive u, o que contraria a suposição que o
  algoritmo acabou de descobrir u. Isso elimina a possibilidade que u
  $\rightarrow$ v seja um arco cruzado pois não é possível que \texttt{d[v]} <
  \texttt{f[v]} < \texttt{d[u]} < \texttt{f[u]}. Analogamente, também não é
  possível que v $\rightarrow$ u seja arco cruzado pois isso implicaria que
  \texttt{d[u]} < \texttt{f[u]} < \texttt{d[v]} < \texttt{f[v]} o que significa
  que u e todos seus atingíveis já foram fechados, o que contraria a suposição
  que o algoritmo ainda está percorrendo u. Contradição.
\end{description}

\section*{Questão 2}
\begin{lstlisting}
static int cnt, pre[maxV], bcnt, low[maxV];
static int parnt[maxV];

void all_bridges (Graph G) {
	Vertex v;
	cnt = bcnt = 0;
	for (v = 0; v < G->V; v++)
		pre[v] = -1;
	for (v = 0; v < G->V; v++)
		if (pre[v] == -1) {
			parnt[v] = v;
			bridgeR(G, v);
		}
	
}

void bridgeR (Graph G, Vertex v) {
	link p; Vertex w;
	pre[v] = cnt++;
	low[v] = pre[v];
	for (p=G->adj[v];p!=NULL;p=p->next)
		if (pre[w=p->w] == -1) {
			parnt[w] = v;
			bridgeR(G, w);
			
			if (low[v] > low[w]) low[v]=low[w];
			if (low[w] == pre[w]) {
				bcnt++;
				printf("%d-%d\n", v, w);
			}
		}
		else if (w!=parnt[v] && low[v]>pre[w])
			low[v] = pre[w];
}
\end{lstlisting}

\section*{Questão 3}
\begin{lstlisting}
/**
GRAPHremoveMultiV recebe como parametros:
 - Um Graph G inicializado e propriamente populado
 - Um vetor removed de flags com tamanho pelo menos G->V
 - Um vetor map Vertex com tamanho pelo menos G->V

Pre-condicoes:
 - G esta inicializado e propriamente populado
 - Para todo vertice v em G vale que removed[v] = 1 caso o vertice v
tenha sido removido e removed[v] = 0 caso contrario.

Pos-condicoes:
 - G nao sofre alteracoes
 - O algoritmo retorna um novo grafo de tamanho G->V - x, sendo x o
numero de vertices removidos
 - O grafo retornado possui todos os vertices nao removidos com nomes
possivelmente trocados
e todas as arestas que nao tinham ponta em um vertice removido.
 - Para todo vertice v de G temos que map[v] = w sendo w o novo nome
do vertice no novo grafo retornado caso v nao tenha sido removido ou
map[v] = -1 caso contrario.*/
Graph GRAPHremoveMultiV(Graph G, int removed[], Vertex map[]){
	Vertex i, count = 0;
	link p;
	Graph G2;
	for(i = 0; i < G->V; i++)
		map[i] = removed[i] ? -1 : count++;
	
	G2 = GRAPHinit(count + 1);
	
	for(i = 0; i < G->V; i++) {
		if(removed[i])
			continue;
		for(p = G->adj[i]; p != NULL; p = p->next)
			if(!removed[p->w])
				DIGRAPHinsertA(G2, map[i], map[p->w]);
	}
	
	return G2;
}
\end{lstlisting}


\end{document}
