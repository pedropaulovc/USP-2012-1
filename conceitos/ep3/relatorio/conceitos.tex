%%%%%%%%%%%%%%%%%%%%%%%%%%%%%%%%%%%%%%%%%%%%%%%%%%%%%%%%%%%%%%%%%%%%%%%%%%%%%%%%
\documentclass[brazil,times]{abnt}

\usepackage[T1]{fontenc}
\usepackage[portuguese]{babel}
\usepackage[utf8]{inputenc}
\usepackage{url}
\usepackage{graphicx}
\usepackage[pdfborder={0 0 0}]{hyperref}
\usepackage{amssymb}
\usepackage{alltt}
\usepackage{epigraph}
\usepackage{listingsutf8}

\makeatletter
\makeatother

\lstset{
	language=Caml,
	tabsize=2,
	inputencoding=utf8,
	basicstyle=\scriptsize,
	showspaces=false,
	showstringspaces=false,
	showtabs=false,
	}

%%%%%%%%%%%%%%%%%%%%%%%%%%%%%%%%%%%%%%%%%%%%%%%%%%%%%%%%%%%%%%%%%%%%%%%%%%%%%%%%
\begin{document}

\autor{Arthur Branco Costa - 7278156\\
      Daniel Paulino Alves - 7156894\\
      Felipe Yamaguti - 7295336\\
      Pedro Paulo Vezzá Campos - 7538743\\
      Thiago Tatsuo Nagaoka - 7289197}

\titulo{SML - Programação Funcional}

\comentario{Terceiro exercício-programa apresentado para avaliação na disciplina MAC0316, do curso de Bacharelado em Ciência da Computação, turma 45, da Universidade de São Paulo, ministrada pela professora Ana Cristina Vieira de Melo.}

\instituicao{Departamento de Ciência da Computação\par
			Instituto de Matemática e Estatística\par
			Universidade de São Paulo}

\local{São Paulo, SP - Brasil}

\data{\today}

\capa
\folhaderosto

\tableofcontents

%%%%%%%%%%%%%%%%%%%%%%%%%%%%%%%%%%%%%%%%%%%%%%%%%%%%%%%%%%%%%%%%%%%%%%%%%%%%%%%%
\chapter*{Introdução}
``{\small It's going to be legen\ldots wait for it, dary!'' \hfill --\emph{Barney Stinson}}

Programação Funcional é um paradigma baseado no $\lambda$-\textit{Calculus}, em oposição ao paradigma imperativo fundamentado na essência da máquina de Turing. Diferente desta última, caracterizada pela representação de estados e suas sucessivas transformações, a programação funcional avalia o problema segundo outra ótica, dando margem a uma interpretação diferenciada.

Ela objetiva a criação de abstrações puramente funcionais, ou seja, mapeamentos de valores de um domínio para outro, sem efeitos colaterais, haja vista que a presença de procedimentos representaria uma violação neste paradigma. Além disso, de acordo com esta concepção, uma vez que o conceito de estado não é aplicável, valores calculados, quando armazenados, o são na forma de constantes, de maneira que o uso extensivo de endereçamentos de memória e o compartilhamento de seus valores associados ao longo do programa (em suma, o emprego de variáveis) são inapropriados.

Este trabalho visa, através da aplicação dos conceitos que dão suporte ao paradigma funcional, criar um sistema de avaliação para a Linguagem Funcional Simplificada (LFSimp), originada a partir da linguagem funcional SML, de forma a manipular diferentes valores e conjuntos de dados e comparar os resultados obtidos pelas principais estratégias de redução existentes: \textit{a priori} e sob demanda.

%%%%%%%%%%%%%%%%%%%%%%%%%%%%%%%%%%%%%%%%%%%%%%%%%%%%%%%%%%%%%%%%%%%%%%%%%%%%%%%%
\chapter{Especificação}

Após considerar a linguagem funcional simplificada (LFSimp) definida no enunciado encontramos dificuldades inicialmente em seguir a sintaxe especificada devido às limitações da linguagem SML, que por ser fortemente tipificada, restringe o uso do domínio e imagem de suas funções. Além disso, por não haver uma função que devolva o tipo da varíavel recebida, a especificação fornecida nos obrigou a encontrar alguma alternativa que satisfizesse o alto grau de polimorfismo requisitado pelo programa. 

Nossa abordagem ficou bastante parecida com a especificação atualizada do problema, na qual foram definidos os tipos básicos do programa e alguns protótipos.

Assim, modificamos partes da LFSimp passada, buscando alterá-la o mínimo possível. É importante frisar que nossa especificação considera que todos os dados de entrada são consistentes. 

A seguir, encontram-se os tipos implementados em nosso EP, com pequenas alterações da nova especificação do enunciado, que serão destacadas adiante:

\begin{verbatim}
datatype Id = a | b | c | d | e | f | g | h | i | j | k | l | m |
			  n | o | p | q | r | s | t | u | v | w | x | y | z;


datatype Expressao
	= Valor of int
	| Bool of bool
	| id of Id
	| Conta of Expressao * string * Expressao
	| IfThenElse of string * Expressao * Expressao * Expressao
	| Aplicacao of  DecFuncao * Expressao list

and DecFuncao = funcao of string * Id list * Expressao;
\end{verbatim}

A primeira diferença foi que consideramos uma letra do alfabeto como sendo uma Id da linguagem e não uma \textit{string}, como especificado anteriormente. Esse fator acarretou um leve efeito colateral no nosso programa: as variáveis não podem ter apenas uma letra em seus nomes.

Além disso, o \texttt{datatype Expressao} consiste da definição do que seria uma expressão na linguagem. Ele é composto pelos seguintes valores atômicos: Valor (um inteiro), Bool (um booleano) e Id, citado acima. É ainda composto por operações, que podem ser: aritméticas, booleanas, IfThenElse e funções.

%TODO falar que não dá pra imprimir o passa-a-passo

%%%%%%%%%%%%%%%%%%%%%%%%%%%%%%%%%%%%%%%%%%%%%%%%%%%%%%%%%%%%%%%%%%%%%%%%%%%%%%%%
\chapter{Estratégias para a sequência de reduções e avaliação dos parâmetros}
Em funções recursivas podem ser utilizadas duas estratégias para a avaliação da expressão final, \textit{a priori} ou sob demanda, variando conforme a linguagem utilizada. Neste trabalho, forçamos a linguagem especificada a realizar ambas estratégias.

\section{\textit{A priori} de ou dentro para fora}
Estratégia de redução mais tradicional nas linguagens de programação, na qual a função é considerada prioritária sob as outras operações, e nada mais é avaliado enquanto seu $\lambda$-termo não é totalmente reduzido. Somente após a função ser totalmente reduzida é que os termos restantes efetuam as respectivas operações.

A implementação dessa estratégia é mais simples que a da anterior, permitindo, assim, a construção de interpretadores e compiladores mais compactos.

%TODO melhorar as frases aqui =]
No trabalho em questão foi implementado o seguinte algoritmo abstrato:

%TODO Potrótipos

\begin{enumerate}
    \item \texttt{apriori = fn : Expressao -> Expressao} \\ é um alias para a função reduz.

    \item \texttt{reduz = fn : Expressao -> Expressao}
        \begin{itemize}
            \item reduz de um valor atômico devolve o próprio valor atômico;
            \item reduz uma conta, invoca a função calcula;
            \item reduz de um ifThenElse, trata a expressão booleana correspondente, tratando-a; e reduz o bloco correspondente à ela.
            \item reduz de aplicação, implica na chamada da função aplicaFuncao
            
        \end{itemize}

     \item \texttt{calcula = fn : Expressao * string * Expressao -> Expressao} \\
      Sua função é reduzir ao máximo cada termo (lados esquerdo e direito) antes de realizar a operação principal;
        Observação: calcula não é responsável por associar Ids a seus respectivos valores, portanto, a função calcula aplicada a uma Id e uma Conta implica na redução apenas da Conta.
        
     \item \texttt{aplicaFuncao = fn : Expressao -> Expressao} \\ é responsável por desmembrar a aplicação, preparando os parâmetros para serem entregues à função evalParametros.
     
     \item \texttt{evalParametros = fn : Id list * Expressao * Expressao list -> Expressao} \\ alias para trocaTudo, que antes, minimiza todas as expressões contidas na lista de parâmetros, através da chamada da função resolveLista
     
     \item \texttt{trocaTudo = fn : Id list * Expressao * Expressao list -> Expressao} \\ é responsável por substituir todos os parâmetros da declaração da função por seus respectivos valores, que serão obtidos a partir da função resolveLista; isso ocorre, através de sucessivas chamadas da função troca;
     
     \item \texttt{troca = fn : Id * Expressao * Expressao -> Expressao} \\ responsável por substituir todas as ocorrências de um dado Id na declaração da função por seu valor reduzido;
     Quando o operando da função troca for uma aplicação, a função trocaLista é chamada; acarretando em uma substituição de todos os parâmetros da lista por seus respectivos valores;
     
     \item \texttt{fn : Id * Expressao list * Expressao -> Expressao list} \\ percebe-se nesta função a avaliação a priori dos parâmetros da função chamada, uma vez que todos eles são substituídos pelos seus respectivos valores, previamente analisados.
     
     \item \texttt{resolveLista = fn : Expressao list -> Expressao list} \\ chama a função reduz para cada elemento da lista de parâmetros.
     
\end{enumerate}
Percebe-se que neste algoritmo, para a realização de qualquer operação, todos os seus operandos são reduzidos primeiramente ao máximo. Isso se aplica tanto para operações aritméticas, booleanas, IfThenElse e para aplicações de funções.


%TODO colar exemplos

\section{Sob Demanda ou de fora para dentro}
Nestes casos, as operações são efetuadas assim que houver uma oportunidade. Assim, a resposta desejada será encontrada logo que a última ocorrência da função for reduzida.

\begin{enumerate}
    \item \texttt{sobdemanda = fn : Expressao -> Expressao} \\ é um alias para a função reduzDemanda.

    \item \texttt{reduzDemanda = fn : Expressao -> Expressao} \\ o comportamento desta função é o mesmo da reduz, excetuando-se o caso quando uma Aplicacao é passada, na qual é chamada a função aplicaFuncaoDemanda;

    
     \item \texttt{calculaDemanda = fn : Expressao * string * Expressao -> Expressao} \\ responsabiliza-se por reordenar a expressão passada, de maneira a permitir o cálculo dos valores assim que tornam-se disponíveis. Por exemplo, uma expressão no formato: 6 + (4 + (2 - (3 + 5) ) ) será convertida na forma seguinte forma: (6 + 4) + (2 - (3 + 5)) e a assim, será realizado o cálculo do primeiro termo, resultando na expressão: 10 + (2 - (3 + 5)); esse processo se repete recursivamente e serão obtidas as seguintes expressões:
     12 - (3 + 5)
     9 - (5)
     4
     
     \item \texttt{aplicaFuncaoDemanda = fn : Expressao -> Expressao} \\ é responsável por desmembrar a aplicação, preparando os parâmetros para serem entregues à função trocaTudo, observa-se que diferentemente da função aplicaFuncao, esta não invoca a resolveLista para a lista de parâmetros, como consequência a aplicação após a chamada de trocaTudo não estará minimizada, abrindo margem para a função calculaDemanda poder calcular os valores sod demanda, apenas nos momentos necessários, ou seja, quando eles serão de fato utilizados.
     
     \item \texttt{trocaTudo = fn : Id list * Expressao * Expressao list -> Expressao} \\ mesma implementação da versão a priori.
     
\end{enumerate}



A seguinte função realiza uma redução sob demanda:
%TODO colar exemplo de função

A estratégia de redução sob demanda não pode ser realizada nos operadores booleanos. Isso ocorre, pois diferente das operações binárias aritméticas, definidas na especificação, nas quais todas as operações (adição e subtração) possuem o mesmo grau de precedência, ao passo que nas operações booleanas os operadores (and e or) possuem uma precedência similar à de soma e multiplicação, por exemplo.

Dessa forma, seria necessária uma verificação de todos os valores pertencentes a expressão antes de efetivamente realizar o cálculo da expressão. Contudo, a natureza da recursão não possibilita tal procedimento.

%%%%%%%%%%%%%%%%%%%%%%%%%%%%%%%%%%%%%%%%%%%%%%%%%%%%%%%%%%%%%%%%%%%%%%%%%%%%%%%%
\chapter{Exemplos}

\begin{lstlisting}
val aa = Valor 7; 
val bb = Valor 3;
val cc = Conta (aa, "-", bb);
val dd = Conta (cc, "-", cc);
val ee = Bool (true);
val ff = Bool (false);
val gg = Conta (ee, "And", ff);
val hh = Conta (ee, "Or", ff);
val ii = Conta (gg, "==", hh);
val pp = Conta (Valor 0, "+", Valor 4);
val mn = Conta (id x, "+", id y);
val pq = Conta (Conta (Valor 9, "-", Valor 9), "+", id x);
val kk = Conta (Valor 0, "+" , Conta (Conta (Conta (Conta (Valor 5, "+", Valor 2), "-", Valor 2), "+", Valor 4), "-", Valor 1));
val testIf = IfThenElse ("if", Conta (Valor 5, "==", Conta (Valor 2, "+", Valor 3)), Valor 4, Valor 3);
val testIfid = IfThenElse ("if", Conta (id a, "==", Conta (Valor 2, "+", id b)), Valor 4, Valor 3);
val teste2If = Conta (Valor 5, "+", IfThenElse ("if", Conta (Valor 5, "==", Valor 4), Valor 5, Valor 0));
val teste5 = Conta (Conta (Valor 7, "-", Valor 8), "-", Conta (Valor 9, "+", Valor 10));
val teste3 = Conta (Bool true, "Or", Conta (Bool true, "And", IfThenElse ("if", Conta (Valor 7, "==", Valor 7), Bool false, Bool false)));
val teste4 = Conta (Conta (Bool false, "Or", Bool false), "Or", Conta (Bool true, "And", Bool false));
val teste2 = Conta (Bool false, "And", Conta (Bool false, "Or", Bool true));


reduz cc;
reduzDemanda cc;
print ("\n\n");
reduz aa;
reduzDemanda aa;
print ("\n\n");
reduz dd;
reduzDemanda dd;
print ("\n\n");
reduz gg;
reduzDemanda gg;
print ("\n\n");
reduz mn;
reduzDemanda mn;
print ("\n\n");
reduz kk;
reduzDemanda kk;

print ("\n\nNovos testes\n\n");

reduz teste5;
reduzDemanda teste5;
print ("\n\n");
reduz teste3;
reduzDemanda teste3;
print ("\n\n");
reduz teste4;
reduzDemanda teste4;
print ("\n\n");
and calcula (id vala, aa, id valc) = Conta (id vala, aa, id valc)
	| calcula (id vala, aa, valc) = Conta (id vala, aa, reduz(valc))
	| calcula (vala, aa, id valc) = Conta (reduz(vala), aa, id valc)
reduz teste2;
reduzDemanda teste2;

print ("\n\nTestes mais novos!\n\n");


val func = funcao ("fun", [x, y, z], Conta(id x, "+", id y));
val aplic = Aplicacao (func, [Valor 5, Conta (Valor 3, "+", Valor 3), Conta (Valor 4, "+", Valor 4)]);

print "\n\n";
reduz  aplic;
reduzDemanda aplic;
print "\n\n";

val bla = funcao ("fun", [a, b], IfThenElse ("if", Conta (id a, "==", id b), id a, id b));
val ble = funcao ("fun", [a, c, d], Conta (Aplicacao (bla, [id a, Valor 0]), "+", id d));

reduz (Aplicacao (ble, [Bool true, Valor 0, Valor 1]));
reduzDemanda (Aplicacao (ble, [Bool true, Valor 0, Valor 1]));

print "\n\n";
\end{lstlisting}

%%%%%%%%%%%%%%%%%%%%%%%%%%%%%%%%%%%%%%%%%%%%%%%%%%%%%%%%%%%%%%%%%%%%%%%%%%%%%%%%
\chapter*{Conclusão}
O desenvolvimento deste projeto evidenciou a importância de avaliadores de linguagens de programação, seja de dentro para fora ou de fora para dentro, e como expressões idênticas, ao ser analisadas sob perspectivas distintas, podem influenciar o
desempenho do programa.

Diferentes estratégias aplicáveis proporcionam diferentes sequências de reduções, sendo esta ou aquela mais apropriada para maximizar a eficiência das reduções a serem efetuadas de acordo com o caso em questão. Logo, não se pode julgar uma dada estratégia melhor do que outra para a maioria dos programas funcionais, mas sim, a que for melhor para um problema (ou uma família de problemas) em particular. Contudo, é notório como as duas opções sempre deverão encontrar resultados válidos, na medida em que satisfazem as propriedades de correção e confluência e garantem a equivalência das expressões finais, variando apenas em eficiência.

Ademais, considerando que o paradigma funcional não é tão usual quanto ao paradigma imperativo, tivemos que nos adaptar à carência de variáveis e comandos iterativos, já que, embora estas ferramentas estejam presentes na linguagem SML, restringimos o nosso arsenal de possibilidades, limitando-nos aos recursos puramente funcionais.

Por fim, é fundamental ressaltar como programar conforme o enfoque funcional constitui uma nova forma de pensar, modelar e abstrair resoluções, não se tratando, portanto, de uma discussão trivial. Assim, torna-se evidente como esta aplicação prática dos conceitos estudados ao longo do semestre foi determinante para a fixação do conhecimento aprendido, tão elaborado, e a sua inegável relevância.

%%%%%%%%%%%%%%%%%%%%%%%%%%%%%%%%%%%%%%%%%%%%%%%%%%%%%%%%%%%%%%%%%%%%%%%%%%%%%%%%
\nocite{*}
\bibliographystyle{abnt-num}
\bibliography{bibliografia}

\end{document}
